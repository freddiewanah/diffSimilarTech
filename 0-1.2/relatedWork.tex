\section{Related Works}
\chen{May need to add state-of-the-art related works like API recommendation}
\subsection{Mining similar software artefects}
Finding similar software artefacts can help developers migrate from one tool to the other which is more suitable to their requirement.
But it is a challenging task to identify similar software artefacts from the existing large pool of candidates.
Therefore, much research effort has been put into this domain.
Different methods has been adopted to mine similar artefacts ranging from high-level software~\cite{mcmillan2012detecting, thung2012detecting}, mobile applications~\cite{chen2015simapp, linares2016automatically}, github projects\cite{zhang2017detecting} to low-level third-party libraries~\cite{teyton2013automatic, chen2016mining, chen2016similartech}, APIs~\cite{gu2017deepam, nguyen2017exploring}, code snippets~\cite{su2016identifying}, or Q\&A questions~\cite{chen2016learning}.
Compared with these research studies, the mined software technologies in this work has much broader scope including not only software-specific artefacts, but also general software concepts (e.g., algorithm, protocol),  tools (e.g., IDE).

\subsection{Extracting opinions about software technologies}
\revise{
Extraction opinions about technologies from sentences using is also important to developers, as it tells them others thoughts on technologies and give them a general vision on the technologies. Using Sentiment Analysis~\cite{pang2008opinion} to solve this type of problem seems to become common, like extracting
sentiments in tweets that are strongly correlate to human
judgment ~\cite{guzman2017exploratory}.
Uddin and Khomh~\cite{uddin2017opiner, uddin2017automatic} extract API opinion sentences in different aspects to show developers' sentiment to that API. Ahasanuzzaman et.al ~\cite{ahasanuzzaman2019caps} classify StackOverflow posts concerning API issues-only with Conditional Random Field (CRF)~\cite{sutton2012introduction}. Lin et.al~\cite{lin2019pattern} produced a pattern-based approach that can identify overall quality, pros, and cons
of APIs. Different from the works mentioned above, our method intended to use sentiment analysis in comparing two similar technologies instead of one single API. And we adopt the the state-of-art BERT model for the classification, which has a higher accuracy on prediction.
}

\subsection{Comparison in Software Engineering}
Given a list of similar technologies, developers may further compare and contrast them for the final selection.
Some researcher investigate such comparison, the comparison is highly domain-specific such as software for traffic simulation~\cite{jones2004traffic}, regression models~\cite{horton2001multiple}, x86 virtualization~\cite{adams2006comparison}, etc.
Michail and Notkin~\cite{michail1999assessing} assess different third-party libraries by matching similar components (such as classes and functions) across similar libraries.
But it can only work for library comparison without the possibility to be extended to other higher/lower-level technologies in Software Engineering.
Instead, we find developers's preference of certain software technologies highly depends on other developers' usage experience and report of similar technology comparisons.
Li et al.~\cite{li2017mining} adopt NLP methods to distill comparative user review about similar mobile Apps.
Different from their works, we first explicitly extract a large pool of comparable technologies.
In addition, apart from extracting comparative sentences, we further organize them into different clusters and represent each cluster with some keywords to help developers understand comparative opinions more easily.
\chen{\cite{de2018library, de2018empirical}}

Finally, it is worth mentioning some related practical projects. 
SimilarWeb~\cite{web:similarweb} is a website that provides both users engagement statistics and similar competitors for websites and mobile applications. 
AlternativeTo~\cite{web:alternativeto} is a social software recommendation website in which users can find alternatives to a given software based on user recommendations. 
SimilarTech~\cite{web:similartechgraph} is a site to recommend analogical third-party libraries across different programming languages.
These websites can help users find similar or alternative websites or software applications without detailed comparison.  